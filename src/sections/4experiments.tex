%! Author = lavandejoey
%! Date = 21/6/2023
%! tex/section/4experiments.tex


\section{实验}\label{sec:experiments}
我们在CIFAR-10数据集上进行了实验。
我们使用的训练集包含50000张32x32的彩色图片,测试集包含10000张图片。
我们的模型在测试集上达到了90\%的准确率。\par
\begin{table}[ht]
    \centering
    \begin{tabular}{cccc}
        \Xhline{1pt}
        \textbf{Model} & \textbf{Train Accuracy} & \textbf{Test Accuracy} & \textbf{Number of parameters} \\
        \Xhline{1pt}
        ResNet50       & 99\%                    & 90\%                   & 25.6M                         \\
        \hline
    \end{tabular}
    \caption{Experiment Results}
    \label{tab:results}
\end{table}

\subsection{实验设置}\label{subsec:exp-setup}
我们的所有实验都在一台配备了16GB内存和一个NVIDIA GTX 1080Ti显卡的电脑上进行。
我们使用Python 3.7和PyTorch 1.4作为编程语言和深度学习框架。
我们使用Adam优化器来训练我们的模型,学习率设置为0.001。\par

\subsection{训练过程}\label{subsec:training}
我们的模型在训练集上进行了100个epoch的训练。
每个epoch之后,我们都会在验证集上评估模型的性能。
模型的参数会在验证集的性能最好的时候保存下来。\par

\subsection{测试结果}\label{subsec:testing}
我们的模型在测试集上达到了90\%的准确率,这比其他常用的神经网络模型的性能要好。
我们的实验结果显示,ResNet50在处理CIFAR-10数据集上具有非常出色的性能。\par

\subsection{进一步的讨论}\label{subsec:discussion}
虽然我们的模型在CIFAR-10数据集上取得了很好的结果,但在其他更复杂的数据集上,比如ImageNet,可能需要更深的网络结构和更复杂的优化策略。
此外,为了提高模型的泛化能力,可能需要引入一些正则化技术,比如Dropout和权重衰减。\par
