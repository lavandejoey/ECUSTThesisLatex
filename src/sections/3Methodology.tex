%! Author = lavandejoey
%! Date = 21/6/2023
%! tex/section/3Methodology.tex


\section{研究方法}\label{sec:methodology}

在我们的研究中,我们使用了深度学习的方法进行图像分类。
我们使用了一个名为ResNet的深度神经网络,该网络通过使用残差学习来解决深度神经网络中的退化问题。\par
\begin{equation}
    F(x) = H(x) - x\label{eq:resnet}
\end{equation}

这里的\(F(x)\)表示残差函数,\(H(x)\)是目标映射,\(x\)是输入。\par
\subsection{ResNet的数学推导}\label{subsec:resnet-derivation}

考虑一个多层神经网络,其目标映射为\(H(x)\),我们可以将其表示为以下形式:

\begin{equation}
    H(x) = F(x) + x\label{eq:resnet-target}
\end{equation}

其中\(F(x)\)是残差函数,\(x\)是输入。
在ResNet中,我们的目标是让\(F(x)\)趋近于零。
通过这种方式,我们将模型的训练目标转化为学习残差函数,从而使得模型更容易被优化。\par
假设我们的网络共有\(L\)层,那么在第\(l\)层,我们的目标映射可以表示为:

\begin{equation}
    H^{(l)}(x) = F^{(l)}(x) + x^{(l-1)}\label{eq:resnet-layer}
\end{equation}

其中,\(x^{(l-1)}\)是第\(l-1\)层的输出,\(F^{(l)}(x)\)是第\(l\)层的残差函数。
通过堆叠多个这样的残差块,我们可以构建出一个深度神经网络。\par
然后我们在代码中实现了这个网络:

\begin{verbatim}
# Here is some Python code
import torch
from torchvision.models import resnet50

model = resnet50(pretrained=True)
\end{verbatim}

\subsection{数据预处理}\label{subsec:data-preprocessing}
在训练模型之前,首先需要对数据进行预处理。
我们的数据来自一个名为ImageNet的大规模视觉数据库,该数据库包含超过一千万的手动标注的图像,涵盖了二千多个类别。\par
数据预处理主要包括图像的裁剪和缩放,以及像素值的归一化。
我们将所有的图像都缩放到224x224的大小,并将像素值归一化到0-1的范围。
这样处理后,可以保证模型在处理不同的图像时,具有统一的输入格式。\par
\begin{verbatim}
# Here is some Python code for data preprocessing
from torchvision import transforms

transform = transforms.Compose([
    transforms.Resize((224, 224)),
    transforms.ToTensor(),
    transforms.Normalize(mean=[0.485, 0.456, 0.406], std=[0.229, 0.224, 0.225]),
])
\end{verbatim}

\subsection{模型训练}\label{subsec:model-training}
模型训练的目标是通过最小化在训练集上的损失,来找到模型的最优参数。
我们使用的是交叉熵损失函数,并使用随机梯度下降(SGD)作为优化算法。
模型的训练过程中,我们采用了早停法来防止过拟合,也就是当验证集上的损失不再下降时,我们就停止模型的训练。\par
\begin{verbatim}
# Here is some Python code for model training
criterion = torch.nn.CrossEntropyLoss()
optimizer = torch.optim.SGD(model.parameters(), lr=0.01)

for epoch in range(100):
    for images, labels in dataloader:
        images = images.to(device)
        labels = labels.to(device)

        outputs = model(images)
        loss = criterion(outputs, labels)

        optimizer.zero_grad()
        loss.backward()
        optimizer.step()
\end{verbatim}

\subsection{模型评估}\label{subsec:model-evaluation}
模型的评估是在独立的测试集上进行的。
我们使用的评估指标是准确率,也就是模型预测正确的样本数占总样本数的比例。
为了更公正地评估模型的性能,我们还使用了混淆矩阵和ROC曲线等工具,以了解模型在不同类别上的性能。\par
\begin{verbatim}
# Here is some Python code for model evaluation
from sklearn.metrics import confusion_matrix, roc_curve

model.eval()
with torch.no_grad():
    predictions = []
    targets = []
    for images, labels in dataloader:
        images = images.to(device)
        labels = labels.to(device)

        outputs = model(images)
        _, predicted = torch.max(outputs, 1)
        predictions.extend(predicted.tolist())
        targets.extend(labels.tolist())

cm = confusion_matrix(targets, predictions)
roc = roc_curve(targets, predictions)
\end{verbatim}
